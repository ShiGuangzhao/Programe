\documentclass[a4paper,11pt,onecolumn,twoside]{article}
\usepackage{ctex}
\usepackage{times}
\usepackage{setspace}
\usepackage{fancyhdr}
\usepackage{graphicx}
\usepackage{subfigure}
\usepackage{multicol}
\usepackage{wrapfig}
\usepackage{array}  
\usepackage{fontspec,xunicode,xltxtra}
\usepackage{titlesec}
\usepackage{titletoc}
\usepackage[titletoc]{appendix}
\usepackage[hyphens]{url}
\usepackage{cite}
\usepackage{listings}
\usepackage[framed,numbered,autolinebreaks,useliterate]{mcode} % 插入代码
\XeTeXlinebreaklocale "zh"
\XeTeXlinebreakskip = 0pt plus 1pt minus 0.1pt

\lstset{basicstyle=\footnotesize\ttfamily,breaklines=true, language=bash, xleftmargin=2em, xrightmargin=1em,  aboveskip=1em}
  

%%%%%%%%%%%%%%%%%%%%%%%%%%%%%%%%%%%%%%%%%%%%%%%%%%%%%%%%%%%%%%%%
%  lengths
%    下面的命令重定义页面边距,使其符合中文刊物习惯。
%%%%%%%%%%%%%%%%%%%%%%%%%%%%%%%%%%%%%%%%%%%%%%%%%%%%%%%%%%%%%%%%
\addtolength{\topmargin}{-54pt}
\setlength{\oddsidemargin}{-0.9cm}  % 3.17cm - 1 inch
\setlength{\evensidemargin}{\oddsidemargin}
\setlength{\textwidth}{17.00cm}
\setlength{\textheight}{24.00cm}    % 24.62




%%%%%%%%%%%%%%%%%%%%%%%%%%%%%%%%%%%%%%%%%%%%%%%%%%%%%%%%%%%%%%%%
% 首页页眉页脚定义
%%%%%%%%%%%%%%%%%%%%%%%%%%%%%%%%%%%%%%%%%%%%%%%%%%%%%%%%%%%%%%%%
% \fancypagestyle{plain}{
% \fancyhf{}
% \lhead{第~XX~卷\quad 第~X~期\\
% \scriptsize{XXXX~年~XX~月}}
% \chead{\centering{X~~X~~X~~期~~刊\\
% \scriptsize{\textbf{The trip to get the Sutra}}}}
% \rhead{Vol. XX, No. XX\\
% \scriptsize{October, 2004}}
% \lfoot{}
% \cfoot{}
% \rfoot{}}

%%%%%%%%%%%%%%%%%%%%%%%%%%%%%%%%%%%%%%%%%%%%%%%%%%%%%%%%%%%%%%%%
% 首页后根据奇偶页不同设置页眉页脚
% R,C,L分别代表左中右,O,E代表奇偶页
%%%%%%%%%%%%%%%%%%%%%%%%%%%%%%%%%%%%%%%%%%%%%%%%%%%%%%%%%%%%%%%%
% \pagestyle{fancy}
% \fancyhf{}
% \fancyhead[RE]{第~XX~卷}
% \fancyhead[CE]{西~~天~~取~~经~~记}
% \fancyhead[LE,RO]{\thepage}
% \fancyhead[CO]{猴~~哥等:王母娘娘寿筵上蟠桃生长过程仿真与分析}
% \fancyhead[LO]{第~X~期}
% \lfoot{}
% \cfoot{}
% \rfoot{}

\makeatletter
\newenvironment{figurehere}
  {\def\@captype{figure}}
  {}
  % \newcommand\figcaption{\def\@captype{figure}\caption}
  % \newcommand\tabcaption{\def\@captype{table}\caption}
\makeatother


%---------------------------------------------------------------------
%	引用文献设置为上标
%---------------------------------------------------------------------
\newcommand{\supercite}[1]{\textsuperscript{\cite{#1}}}

