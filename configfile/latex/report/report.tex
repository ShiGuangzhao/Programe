\documentclass[a4paper,11pt,onecolumn,twoside]{article}
\usepackage{ctex}
\usepackage{times}
\usepackage{setspace}
\usepackage{fancyhdr}
\usepackage{graphicx}
\usepackage{subfigure}
\usepackage{wrapfig}
\usepackage{array}  
\usepackage{fontspec,xunicode,xltxtra}
\usepackage{titlesec}
\usepackage{titletoc}
\usepackage[titletoc]{appendix}
\usepackage[hyphens]{url}
\usepackage{cite}
\usepackage{listings}

%%%%%%%%%%%%%%%%%%%%%%%%%%%%%%%%%%%%%%%%%%%%%%%%%%%%%%%%%%%%%%%%
%  lengths
%    下面的命令重定义页面边距,使其符合中文刊物习惯。
%%%%%%%%%%%%%%%%%%%%%%%%%%%%%%%%%%%%%%%%%%%%%%%%%%%%%%%%%%%%%%%%
\addtolength{\topmargin}{-54pt}
\setlength{\oddsidemargin}{-0.9cm}  % 3.17cm - 1 inch
\setlength{\evensidemargin}{\oddsidemargin}
\setlength{\textwidth}{17.00cm}
\setlength{\textheight}{24.00cm}    % 24.62


\makeatletter
\newenvironment{figurehere}
  {\def\@captype{figure}}
  {}
\makeatother

%---------------------------------------------------------------------
%	引用文献设置为上标
%---------------------------------------------------------------------
\newcommand{\supercite}[1]{\textsuperscript{\cite{#1}}}



\begin{document}

\begin{titlepage}
	\begin{center}
		
    \vspace{4in}
    \phantom{anything}
        \textbf{\zihao{2}\heiti{任务一:最长密码}}\\
        \vspace{0.3in}
        \textbf{\zihao{3}\heiti{课程90:2015年Instant Prime挑战赛任务}}\\
    
	\vspace{\fill}
	
\setlength{\extrarowheight}{3mm}
{\songti\zihao{-3}	
\begin{tabular}{cc}
	
    {\makebox[6\ccwd][s]{项目来源:}}&~{软件工程导论课程}\\
    {\makebox[6\ccwd][s]{姓\qquad\quad 名:}}&~\kaishu{石广钊}\\ 
    {\makebox[6\ccwd][s]{学\qquad\quad 号:}}&~\kaishu{161180111}\\
    {\makebox[6\ccwd][s]{指导老师:}}&~\kaishu{方\qquad 晖}\\

\end{tabular}
 }\\[2cm]
	\end{center}	
\end{titlepage}

\tableofcontents % 生成目录
\clearpage


% \begin{center}
    {\heiti\zihao{2}\textbf{实验三\quad 复阻抗测量实验报告}}\\  
    {\centerline{\kaishu\zihao{5} (电子信息科学与技术 \qquad 石广钊 \qquad 161180111)}}  
\end{center}


\begin{spacing}{1.5}
\songti\zihao{-4}
\setcounter{page}{1}
    \section{项目简介}
    \subsection{背景}
    本项目为南京大学电子学院《软件工程导学》课程作业。项目来源于为Codility.com网站 Lesson 90: Tasks from Indeed Prime 2015 challenge中的任务1。 
    项目要求写一个程序实现题目中要求的功能,编程语言不限,且本题对时间复杂读没有要求,只需保证正确性即可。

    \subsection{目的}
    项目要求设计一个函数,对于输入字符串S,返回其中符合密码要求的最长word的长度。其中不同word之间以空格分割,word可作为密码格式需满足三个条件,即:
    \begin{enumerate} [(1)]
            \item 仅包含字母或数字字符(A\-Z, a\-z, 0\-9)。
            \item 需要包含偶数个字母(0,2,4,..)。
            \item 需要包含奇数个数字(1,3,5,..)。
    \end{enumerate}

    实现该功能的编程语言不限,字符串S长度N长度在[1..200]之间。
    
    \section{范围}
    本项目中,由于Python中已经有了具有切片等功能的函数,我们使用Python进行程序的编写,需要实现一个函数:

\begin{lstlisting}{language=python}
def solution(S)
\end{lstlisting}

    编写所得程序可以先通过自己编写的程序进行测试,再通过 codility.com 网站进行测试。

    \section{规格说明}
    \subsection{设计要求}
    \subsection{设计标准}
    \subsection{成果交付}
    \section{方法}

\end{spacing}
\end{document}

